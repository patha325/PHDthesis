\chapter{Neutrino interaction studies}
\label{c:neutrino}

Lab report again, trying to estimate cross-section of interactions. Why? Need it to improve oscillation measurements. Using simulations of muon neutrinos from simulation. What is done, what cuts? How Do we calculate cross-sections? Angular acceptance, look at what Marc has done, do similar things and further the calculations.

Discuss what to expect, what kind of interactions could happen in the TASD/WAGASCI? What could be seen? What information? Baby MIND can also have interactions, which ones would be visible? Can reconstruct muons in Baby MIND, electrons and pions visible and potentially reconstructable, but not now.

\section{Muon charge current quasi-elastic}

Acceptance study, detector layout. Beam settings? Add charge reconstruction, momentum reconstruction. Particle ID? also need interaction ID from TMVA.

Different size gap, different detector layout both early/testbeam and new at Japan.


Cross-section studies. Look at what Marc has done! 

Discuss with paul what and how to do cross-section measurements.

\subsection{Neutrino interactions in TASD + Baby MIND}

\subsubsection{Interactions in TASD}

\subsubsection{Interactions in Baby MIND}

\subsection{Neutrino interactions in WAGASCI + Baby MIND}

\subsubsection{Interactions in Baby MIND}

\subsubsection{Interactions in the ful WAGASCI}

\subsubsection{Simulations vs data}


%\subsection{GAr + BabyMIND}
%\subsubsection{Analysis Simulations}
%\subsection{TASD + BabyMIND}
%\subsubsection{Analysis Simulations}
%\subsection{Full WAGASCI}
%\subsubsection{Analysis with Simulations}
%\subsubsection{Simulations vs data}

\section{Electron charge current quasi-elastic}

