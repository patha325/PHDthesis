\vspace*{0.75in}
\begin{center} {\bf Abstract}\end{center}

The T2K long-baseline neutrino experiment in Japan is designed to study neutrino oscillations, to determine the mixing angles and mass-squared difference of the neutrino mass eigenstates and, potentially, to discover CP violation in neutrinos by comparing neutrino to antineutrino oscillations. In the near detector complex 280 m downstream of the production target at the Japanese Particle Accelerator Research Centre (J-PARC), the WAGASCI experiment will measure the ratio of cross sections from neutrinos interacting with a water and scintillator targets, in order to constrain neutrino cross sections essential for the T2K neutrino oscillation measurements. A prototype Magnetised Iron Neutrino Detector, called Baby MIND, has been constructed at CERN and will act as a magnetic spectrometer behind the main WAGASCI target. The Baby MIND spectrometer was installed between February and March 2018 in the near detector complex, behind WAGASCI and is able to measure the charge and momentum of the outgoing muon from neutrino charged current interactions inside the WAGASCI target, to be able to perform full neutrino event reconstruction. Baby MIND collected data in the reverse horn focussed antineutrino beam between April and May 2018. In this thesis, the Baby MIND spectrometer is described in detail along with the performance from initial beam tests performed with the Proton Synchrotron (PS) charged particle beam at the T9 test beam facility at CERN. The test beam was used to perform measurements of track reconstruction efficiency and charge reconstruction efficiency, using dedicated reconstruction programmes, SaRoMaN and SAURON. The software environment used to perform event reconstruction in the complex detector geometry of Baby MIND is described in this thesis. Furthermore, a machine learning multi-variate analysis was used to perform particle identification between muons and hadrons, allowing for a pure selection of muons in the test beam. NuSTORM is a novel type of neutrino beam from the decay of muons in a storage ring. This type of facility produces well defined beams of $\nu_\mu$ and $\bar{\nu}_e$ neutrinos. A study is performed in the thesis to determine the expected sensitivity of measuring neutrino interactions in a fully active scintillator neutrino target, with a magnetised iron detector downstream. This analysis also benefited from an identification of the different event types by using a machine learning multi-variate approach. Finally, results are presented on charged current quasi-elastic neutrino and antineutrino interactions in iron reconstructed with the Baby MIND detector during the 2018 neutrino data taking at J-PARC.

\if{0}
The WAGASCI and Baby MIND detectors have been installed at the J- PARC neutrino beam line where they will measure the difference in cross sec- tions from neutrinos interacting with a water and scintillator targets. 
The main goal is to constrain neutrino cross sections, essential for the T2K neutrino os- cillation measurements. 
The Baby MIND (Magnetised Iron Neutrino Detector) spectrometer was installed between February and March 2018 in the B2 under- ground neutrino area at J-PARC with an off-axis angle of 2.5o. 


Baby MIND will be able to measure the charge and momentum of the outgoing muon from neutrino charged current interactions inside the WAGASCI target, to be able to perform full neutrino event reconstruction.


The WAGASCI experiment being built at the J-PARC neutrino beam line will measure the ratio of cross sections from neutrinos interacting with a water and scintillator targets, in order to constrain neutrino cross sections, essential for the T2K neutrino oscillation measurements. 
A prototype Magnetised Iron Neutrino Detector (MIND), called Baby MIND, has been constructed at CERN and will act as a magnetic spectrometer behind the main WAGASCI target. 
Baby MIND will be installed inside the WAGASCI cavern at J-PARC in the beginning of 2018. Baby MIND will be able to measure the charge and momentum of the outgoing muon from neutrino charged current interactions, to enable full neutrino event reconstruction in WAGASCI.
During the summer of 2017, Baby MIND was operated and characterised at the 
Results from this test beam will be presented, including charge identification performance and momentum resolution for charged tracks. These results will be compared to the Monte Carlo simulations. 
Finally, simulations of charge-current quasi-elastic (CCQE) neutrino interactions in an active scintillator neutrino target, followed by the Baby MIND spectrometer, will be shown to demonstrate the capability of this detector set-up to perform cross-section measurements under different assumptions.


T2K (Tokai-to-Kamioka) is a long-baseline neutrino experiment in Japan designed to study various parameters of neutrino oscillations. 

A near detector complex (ND280) is located 280 m downstream of the production target and measures neutrino beam parameters before any oscillations occur.
 ND280's measurements are used to predict the number and spectra of neutrinos in the Super-Kamiokande detector at the distance of 295 km. 
 The difference in the target material between the far (water) and near (scintillator, hydrocarbon) detectors leads to the main non-cancelling systematic uncertainty for the oscillation analysis. 
 In order to reduce this uncertainty a new WAter-Grid-And-SCintillator detector (WAGASCI) has been developed. 
 A magnetized iron neutrino detector (Baby MIND) will be used to measure momentum and charge identification of the outgoing muons from charged current interactions. 
 The Baby MIND modules are composed of magnetized iron plates and long plastic scintillator bars read out at the both ends with wavelength shifting fibers and silicon photomultipliers. 
 The front-end electronics board has been developed to perform the readout and digitization of the signals from the scintillator bars. Detector elements were tested with cosmic rays and in the PS beam at CERN. 
 The obtained results are presented in this paper.


The WAGASCI experiment being built at the J-PARC neutrino beam line will measure the difference in cross sections from neutrinos interacting with a water and scintillator targets, in order to constrain neutrino cross sections, essential for the T2K neutrino oscillation measurements. A prototype Magnetised Iron Neutrino Detector (MIND), called Baby MIND, is being constructed at CERN to act as a magnetic spectrometer behind the main WAGASCI target to be able to measure the charge and momentum of the outgoing muon from neutrino charged current interactions.


The LHC at CERN is now undergoing a set of upgrades to increase the center of mass energy for the colliding particles to be able to explore new physical processes. The focus of this thesis lies on the so called phase II upgrade which will preliminarily be completed in 2023. After the upgrade the LHC will be able to accelerate proton beams to such a velocity that each proton has a center of mass energy of 14 TeV. One disadvantage of the upgrade is that it will be harder for the atlas detector to isolate unique particle collisions since more and more collisions will occur simultaneously, so called pile-up. For 14 TeV there does not exist a full simulation of the atlas detector. This thesis instead uses data from Monte Carlo simulations for the particle collisions and then uses so called smearing functions to emulate the detector responses. This thesis focuses on how a mono-jet analysis looking for different wimp models of dark matter will be affected by this increase in pile-up rate. The signal models which are in focus are those which try to explain dark matter without adding new theories to the standard model or QFT, such as the effective theory D5 operator and light vector mediator models. The exclusion limits set for the D5 operators mass suppression scale at 14 TeV and 1000 fb-1are 2-3 times better than previous results at 8 TeV and 10 fb-1. For the first time limits have been set on which vector mediator mass models can be excluded at 14 TeV.
\fi


%This is a dissertation outline using the style guidelines defined by the University of Glasgow.

