\chapter{Testbeam}
\label{c:Testbeam}

\section{Testbeam}


Baby MIND was exposed to the T9 test beam in 2016 to perform an electronics validation using a TASD, (Totally Active Scintillating Detector) [6] and in 2017 to commission with the full Baby MIND. The T9 test beam deries particles from the PS synchrotron at CERN, which produces both hadron (pion) and muon beams between 0.5 and 10 GeV/c.

Generic set-up. Where was this done? Etc. etc.

Explain T9, PS, beam settings etc.

\section{TASD-testbeam}
\subsection{Setup}
Different datasets, beam intensities etc. etc.
\subsection{Results}
What were the results, showed that the software seems to work, both digitisation and reconstruction. Anything else? All the electronics etc. To few planes. No magnetic field.

\section{MIND-testbeam}
\subsection{Setup}
Will we have different datasets with different detector layouts? 
\subsection{Results}

Initial results, with pion contamination and initial unpacking. Discuss issue with cuts. Assuming that all hits had hit amplitude, unpacking could not collate hits correctly.

Confirm reconstruction even further. Will be able to properly run the code on the data.

Charge ID study momentum reconstruction. look at charge id for particles correctly momentum reconstructed, look after TMVA pID. 


Add details, not understanding what is correct data, further study has been performed but not solved problem efficiently. Unknown beam composition.


