\chapter{Conclusions and Outlook}
\label{c:conclusion}

\section{Conclusions}

The work performed in this thesis work has performed all of the goals set out at the start and beyond by providing two independent software frameworks for simulation and reconstruction of muon tracks in the Baby MIND. Further more work has been performed with the construction and commissioning of the detector. This work is seen as part of a larger chain which is still ongoing and benefits from completed and partial results that have been detailed in this thesis.

\pagebreak
\newpage
\section{Outlook}

%\subsection{Baby MIND and WAGASCI}

\subsection{Software}

During this thesis work two software frameworks have been developed. SaRoMaN was expanded to be able to be used for the Baby MIND and SAURON was developed from scratch to take advantage of lessons learned as well as set a standard for future development. A simple step in improvement required for both frameworks would be more input from other parts of the collaboration. In either framework there is a lot of detector descriptions and DAQ which could be handled in better and faster ways with more information. These are required to improve results and to provide simulations closer to data.

%\section{Project management}

\subsection{Machine learning in physics}

A large part of particle physics and physics in general is understanding collected data and performing data analysis. This is often based on looking for underlying physical models and comparing between theory and experiments not only to test a theory but also to understand the collected data. Machine learning provides a new method where pattern recognition is performed with only some naive assumptions about the underlying physics. This allows for patterns, which do not come from directly from the underlying physics to be found and used for further analysis.

\subsection{Neutrino physics}

There is a lot of interesting physics being performed to understand the parameters of neutrino oscillations as well as also finding a complete theory to describe how neutrinos have mass. Aside from this there are interesting experiments where muon detectors are being used for tomography. One example of this is trying to find chambers in the pyramids which provides a non-intrusive way of searching~\cite{86Morishima}. Another interesting aspect comes from doing precise measurements of nuclear reactors, to be used for non-proliferation~\cite{87Askins} and perhaps even to search for nuclear submarines in a noisy environment where conventional techniques may not work~\cite{88Jocher}.


