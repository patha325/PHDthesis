\chapter{Novel software techniques for neutrino physics}
\label{c:test}

In the field of neutrino physics machine learning  is not often used, some plots for dune etc etc. This was used in this collaboration using TMVA for particle identification/particle selections to chose muons in a background sample for the test beam.
For the reconstruction, kalman filtering and cellular automaton for few hit events, are used as well.
Outside of this novel software frameworks such as virtual machines, docker containers and continuous integration techniques have been used.

\section{Machine learning}

What is machine learning? How is it used today? Techniques for particle physics etc. novel ML techniques.

\subsection{Machine learning in particle physics}

\subsubsection{Machine learning applied to charge identification}

Show TMVA plots, comparing normal likelihood techniques to machine learning, different applications and different methods. Explain the different methods.

Relating more to data science than expected results from physics.

\subsubsection{Machine learning applied to event identification}

\section{Containers}
One of the main issues with scientific software is that they are developed by small teams, often not made publicly available and often the knowledge is only spread through scientific papers. This means that there are many great software solutions and packages which are not used by scientific collaborations, or are misused due to a lack of knowledge. One could argue that code is easy to read and that the purpose of a software should be easy to understand, however this assumes that all developers follows some form of coding standard and has/takes the time to develop their code properly which is often not the case. The clear solution to this problem would be to let physicists spend more time documenting and cleaning up their code as well as sharing it publicly. One problem which may arise is the use of specific third party software versions which make installation quite difficult especially in a climate where documentation is not prioritised and where software is constantly in development.

One proposed solutions which solves the problem would be to provide a full runnable coding platform such as a virtual machine. The benefits are numerous however most of these are even improved by moving to a new software technique known as a container.

\subsection{Docker vs Virtual machine}

\subsection{Container for particle physics simulations}
Containing root and geant.

\subsection{Container for neutrino physics simulations}
The above with genie

\subsection{Container for Baby MIND}
The above with SaRoMaN (Can not be done due to missing software.) and with new software. 

\section{Continuous integration}
Does this need to mentioned? It is a good technique which may or may not be used? For everything.

\section{Other good practises}

\subsection{Versioning}
Used for Container and data (gitlab etc). 


\subsection{Testing}
Installation etc. Ensuring that software 

\subsection{Documenting}
